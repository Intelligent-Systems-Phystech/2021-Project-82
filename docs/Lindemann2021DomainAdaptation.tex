\documentclass[12pt, twoside]{article}
\usepackage{jmlda}
\newcommand{\hdir}{.}

\begin{document}

\title
    {Обучение с экспертом для выборки со многими доменами} % Название
\author
    {Н.\,А.~Линдеманн, А.\,В.~Грабовой} % основной список авторов, выводимый в оглавление
\email
    {lindemann.na@phystech.edu; andriy.graboviy@mail.ru}
%\thanks
%    {Работа выполнена при
%     %частичной
%     финансовой поддержке РФФИ, проекты \No\ \No 00-00-00000 и 00-00-00001.}
%\organization
%    {$^1$Организация, адрес; $^2$Организация, адрес}
\abstract
    {
    
Рассматривается задача аппроксимации multi-domain выборки единой мультимоделью -- смесью экспертов. В качестве данных предполагается использовать выборку, которая содержит в себе несколько доменов. Метка домена для каждого объекта отсутствует. Каждый домен аппроксимируется локальной моделью. В работе рассматривается двухэтапная задача оптимизации на основе ЕМ-алгоритма.
    
В качестве данных используются выборки отзывов сайта Amazon для разных типов товара. В качестве локальной модели предполагается использовать линейную модель, а в качестве признакового описания отзывов использовать tf-idf вектора внутри каждого домена.
	
\bigskip
\noindent
\textbf{Ключевые слова}: \emph {}

}

%данные поля заполняются редакцией журнала
%\doi{10.21469/22233792}
%\receivedRus{01.01.2017}
%\receivedEng{January 01, 2017}

\maketitle
\linenumbers

\section{Введение}
to be done


%%%% если имеется doi цитируемого источника, необходимо его указать, см. пример в \bibitem{article}
%%%% DOI публикации, зарегистрированной в системе Crossref, можно получить по адресу http://www.crossref.org/guestquery/
\begin{thebibliography}{99}


 \bibitem{article}
    \BibAuthor{A.~Nithya, C.~Lakshmi}
   Iris Recognition Techniques: A Literature Survey~//
    \BibJournal{International Journal of Applied Engineering Research}, 2015

 \bibitem{article}
    \BibAuthor{K.~Bowyer, K.~Hollingsworth, and P.~Flynn}
   Image Understanding for Iris Biometrics: A Survey~//
    \BibJournal{Computer Vision and Image Understanding}, 2008. Vol.~110. \No\,2. pp.~281--307
	
 \bibitem{article} 
    \BibAuthor{K.\,A.~Gankin, A.\,N.~Gneushev, and I.\,A.~Matveev}
   Iris image segmentation based on approximate methods
with subsequent refinements~//
    \BibJournal{Journal of Computer and Systems Sciences International}, 2014. Vol.~53. \No\,2. pp.~224--238.
	\BibDoi{10.1134/S1064230714020099}.
	
  \bibitem{article}
    \BibAuthor{I.\,A.~Matveev}
   Detection of iris in image by interrelated maxima of brightness gradient projections~//
    \BibJournal{Appl. Comput. Math.}, 2010. Vol.~9. \No\,2. pp.~252--257.
	
 
 	
\end{thebibliography}

%%%% если имеется doi цитируемого источника, необходимо его указать, см. пример в \bibitem{article}
%%%% DOI публикации, зарегистрированной в системе Crossref, можно получить по адресу http://www.crossref.org/guestquery/.

\end{document}